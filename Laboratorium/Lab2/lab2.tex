\documentclass[a4paper,12pt]{article}
\usepackage{polski}
\usepackage[utf8]{inputenc}
\usepackage[OT4]{fontenc}
\usepackage{mathtools}
\usepackage{float}
\usepackage{graphicx}
\usepackage{multirow}
\usepackage{listings}
\usepackage{xcolor}

\newcommand{\h}[1]{\noindent \bf #1 \rm \\ \noindent}
\newcommand{\italic}[1]{\it #1 \rm}

\lstdefinestyle{mystyle}{
	backgroundcolor=\color{lightgray},   
	commentstyle=\color{teal},
	keywordstyle=\color{blue},
	numberstyle=\tiny,
	stringstyle=\color{purple},
	basicstyle=\ttfamily\footnotesize,
	breakatwhitespace=false,         
	breaklines=true,                 
	captionpos=b,                    
	keepspaces=true,                 
	numbers=left,                    
	numbersep=5pt,                  
	showspaces=false,                
	showstringspaces=false,
	showtabs=false,                  
	tabsize=2,
}
\lstset{style=mystyle}

\begin{document}
	
\begin{center}
	\LARGE
	Bazy Danych \\
	\large
	LABORATORIUM 2 
\end{center}
\vspace{1cm}	
	
\h{Funkcja STDDEV:}
Funkcja STDDEV służy do obliczania odchylenia standardowego wybranych komórek z danej kolumny.
\begin{lstlisting}[language=SQL]
SELECT STDDEV(<kolumna>) FROM <nazwa tabeli>;
\end{lstlisting}
\vspace{5mm}

\h{Funkcja VARIANCE:}
Funkcja VARIANCE służy do obliczania wariancji wybranych komórek z danej kolumny.
\begin{lstlisting}[language=SQL]
SELECT VARIANCE(<kolumna>) FROM <tabela>;
\end{lstlisting}
\vspace{5mm}	

\h{Klauzula GROUP BY:}
Klauzula GROUP BY może zostać użyta do grupowania rekordów według wartości poszczególnych kolumny/kolumn. Pozwala to na wyliczanie wyrażeń agregujących dla określonych grup w danej tabeli.
\begin{lstlisting}[language=SQL]
-- Wyliczenie sredniej pensji wedlug menedzera i departamentu, sortowane rosnaco po sredniej pensji
SELECT 
	manager_id 	  AS "Menedzer", 
	department_id AS "Departament", 
	AVG(salary)   AS "Srednia pensja"
FROM 
	employees
GROUP BY
	manager_id,
	department_id
ORDER BY
	AVG(salary);
\end{lstlisting}
\vspace{5mm}

\newpage
\h{Klauzula HAVING:}
Klauzula HAVING jest klauzulą warunkową dla wyników funkcji agregujących. Często jest używana wraz z GROUP BY. Nie można użyć HAVING dla zwykłych kolumn, a WHERE do kolumn będących wynikami grupowania.
\begin{lstlisting}[language=SQL]
-- Wyliczenie sredniej pensji pracownikow, zarabiajacych ponizej 3000, mniejszej niz 5000, wedlug menedzera i departamentu, sortowane rosnaco po sredniej pensji
SELECT 
	manager_id 	  AS "Menedzer", 
	department_id AS "Departament", 
	AVG(salary)   AS "Srednia pensja"
FROM 
	employees
WHERE
	salary < 3000
GROUP BY
	manager_id,
	department_id
HAVING
	AVG(salary) < 5000
ORDER BY
	AVG(salary);
\end{lstlisting}
\vspace{5mm}

\h{Funkcja SYSDATE:}
Funkcja SYSDATE zwraca aktualną datę systemową. Wymagane jest jednak podanie tabeli, aby się do niej dostać. Przykładowo można użyć tabeli dummy o nazwie DUAL.
\begin{lstlisting}[language=SQL]
SELECT SYSDATE FROM <tabela>; -- Pobranie daty systemowej
\end{lstlisting}
\vspace{5mm}

\h{Funkcja TO\_DATE}
Funkcja TO\_DATE konwertuje zadany ciąg znaków o zadanym formacie na datę. 
\begin{lstlisting}[language=SQL]
SELECT 
	TO_DATE(
		<ciag znakow>, 
		<format daty>
	)
FROM
	<tabela>;
\end{lstlisting}
\vspace{5mm}

\newpage
\h{Funkcja ADD\_MONTHS:}
Funkcja ADD\_MONTHS dodaje do zadanej daty zadaną liczbę miesięcy.
\begin{lstlisting}[language=SQL]
SELECT
	ADD_MONTHS(
		<data>,
		<ilosc miesiecy>
	)
FROM
	tabela;
\end{lstlisting}
\vspace{5mm}

\h{Funkcja MONTHS\_BETWEEN:}
Funkcja obliczająca różnicę między dwoma datami w miesiącach.
\begin{lstlisting}[language=SQL]
SELECT
	MONTHS_BETWEEN(
		<data 1>,
		<data 2>
	)
FROM
	<tabela>;
\end{lstlisting}
\vspace{5mm}

\h{Funkcja EXTRACT:}
Funkcja pozwalająca na wyciągnięcie roku, dnia lub miesiąca z daty.
\begin{lstlisting}[language=SQL]
SELECT EXTRACT(DAY FROM <data>) FROM dual; -- wyjmij dzien
SELECT EXTRACT(MONTH FROM <data>) FROM dual; -- wyjmij mies.
SELECT EXTRACT(YEAR FROM <data>) FROM dual; -- wyjmij rok
\end{lstlisting}
\vspace{5mm}

\h{Funkcja TO\_CHAR:}
Funkcja konwertująca zadaną datę na ciąg znaków o zadanym formacie.
\begin{lstlisting}[language=SQL]
SELECT 
	TO_CHAR(
		<data>,
		<format>
	)
FROM
	<tabela>;
\end{lstlisting}
\vspace{5mm}

\newpage
\h{Funkcja TRUNC:}
Obcina datę do pierwszego dnia wskazanej jednostki czasu. Na przykład pierwszego dnia miesiąca, roku, dnia, tygodnia, kwartału.
\begin{lstlisting}[language=SQL]
SELECT
	TRUNC(
		<data>,
		<ograniczenie>
	)
FROM
	<tabela>;
\end{lstlisting}
\vspace{5mm}

\h{Funkcja LAST\_DAY:}
Funkcja zwracająca ostatni dzień miesiąca, obecnego we wskazanej dacie.
\begin{lstlisting}[language=SQL]
SELECT
	LAST_DAY(<data>)
FROM
	<tabela>;
\end{lstlisting}
\vspace{5mm}

\h{Funkcja NEXT\_DAY:}
Funkcja zwracająca datę, w której wystąpi najbliższy zadany dzień tygodnia (względem zadanej daty).
\begin{lstlisting}[language=SQL]
-- Zwraca date najblizszego dnia tygodnia od zadanej daty
SELECT
	NEXT_DAY(
		<data>,
		<numer dnia tygodnia>
	)
FROM
	<tabela>;
\end{lstlisting}
\vspace{5mm}

\h{Arytmetyka dat:}
Do daty można dodać lub odjąć stałą, aby dodać lub objąć od/do niej dni.
\begin{lstlisting}[language=SQL]
-- Zmiana daty o pewna ilosc dni
SELECT <data>+/-<stala> FROM <tabela>;
\end{lstlisting}
\vspace{5mm}

\noindent
Można obliczyć różnicę między datami w dniach za pomocą operacji odejmowania.
\begin{lstlisting}[language=SQL]
-- Zmiana daty o pewna ilosc dni
SELECT <data> - <data> FROM DUAL
\end{lstlisting}
\vspace{5mm}

\newpage
\h{Operator IS NULL:}
Operator służący do określania czy wartość w kolumnie jest null-em.
\begin{lstlisting}[language=SQL]
SELECT <kolumny> FROM <tabela> WHERE IS NULL;
\end{lstlisting}
\vspace{5mm}

\h{Operator IS NOT NULL:}
Operator służący do określania czy wartość w kolumnie nie jest NULL-em.
\begin{lstlisting}[language=SQL]
	SELECT <kolumny> FROM <tabela> WHERE IS NOT NULL;
\end{lstlisting}
\vspace{5mm}

\h{Funkcja COALESCE:}
Funkcja podmieniająca występujące w zadanych kolumnach wartości NULL na wartość pierwszej z kolumn (albo stałych) z zadanego zbioru, która nie jest NULL-em.
\begin{lstlisting}[language=SQL]
-- Gdy manager_id bedzie NULL to zastap go salary badz na odwrot
SELECT COALESCE(manager_id, salary) FROM employeees;
\end{lstlisting}
\vspace{5mm}

\h{Funkcja NULLIF:}
Funkcja zwracająca NULL, gdy dwie zadane w niej wartości są równe. W przeciwnym wypadku zwraca wartość pierwszą.
\begin{lstlisting}[language=SQL]
SELECT  <kolumny>
FROM    <tabela>
WHERE   NULLIF(<kolumna/kolumna>, <kolumna/wartosc>) IS NULL;
\end{lstlisting}
\vspace{5mm}

\h{Porównania z NULL-em:}
Porównania z zawierające NULL zawsze dadzą wynik NULL.\\

\h{Operacje arytmetyczne z NULL-em:}
Operacje arytmetyczne, w których co najmniej jeden argument jest NULL-em zwracają NULL.\\

\h{Funkcje agregujące z NULL-em:}
Funkcje agregujące po prostu ignorują komórki zawierające wartości NULL. Przykładowo funkcja COUNT nie zlicza takich komórek.\\

\h{Funkcje logiczne z NULL-em:}
Jeżeli argumentem funkcji logicznej jest NULL to zwraca ona NULL.\\

\newpage
\h{Funkcja NVL:}
Funkcja podmienia wartość pierwszego argumentu na drugi jeżeli pierwszy jest NULL-em.
\begin{lstlisting}[language=SQL]
SELECT 
	NVL(<kolumna/wartosc>, <kolumna/wartosc>) 
FROM <tabela>;
\end{lstlisting}
\vspace{5mm}

\h{Funkcja NVL2:}
Funkcja podmienia wartość pierwszego argumentu na drugi jeżeli pierwszy nie jest NULL-em lub na trzeci gdy pierwszy jest NULL-em.
\begin{lstlisting}[language=SQL]
SELECT 
	NVL2(
		<kolumna/wartosc>, 
		<wartosc gdy pierwszy nie NULL>,
		<wartosc gdy pierwszy NULL>
	) 
FROM <tabela>;
\end{lstlisting}
\vspace{5mm}
	
\end{document}