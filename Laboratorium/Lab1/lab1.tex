\documentclass[a4paper,12pt]{article}
\usepackage{polski}
\usepackage[utf8]{inputenc}
\usepackage[OT4]{fontenc}
\usepackage{mathtools}
\usepackage{float}
\usepackage{graphicx}
\usepackage{multirow}
\usepackage{listings}
\usepackage{xcolor}

\newcommand{\h}[1]{\noindent \bf #1 \rm \\ \noindent}
\newcommand{\italic}[1]{\it #1 \rm}

\lstdefinestyle{mystyle}{
	backgroundcolor=\color{lightgray},   
	commentstyle=\color{teal},
	keywordstyle=\color{blue},
	numberstyle=\tiny,
	stringstyle=\color{codepurple},
	basicstyle=\ttfamily\footnotesize,
	breakatwhitespace=false,         
	breaklines=true,                 
	captionpos=b,                    
	keepspaces=true,                 
	numbers=left,                    
	numbersep=5pt,                  
	showspaces=false,                
	showstringspaces=false,
	showtabs=false,                  
	tabsize=2,
}
\lstset{style=mystyle}

\begin{document}

\begin{center}
	\LARGE
	Bazy Danych \\
	\large
	LABORATORIUM 1 
\end{center}
\vspace{1cm}

\h{Operatory warunkowe:}
W SQL mamy do dyspozycji operatory warunkowe:
\begin{itemize}
	\item $=$ - równości
	\item $!=$ lub $<>$ - nierówności
	\item $>$ -  ostrej większości
	\item $>=$ - nieostrej większości
	\item $<$ - ostrej mniejszości
	\item $<=$ - nieostrej mniejszości 
	\item $BETWEEN$ $<wartosc>$ $AND$ $<wartosc>$ - przynależności do przedziału
	\item $LIKE$ $<wzorzec>$ - spełnianie wzorca
	\item $IN$ $(<wartosc>, <wartosc>, ...)$ - przynależności do zadanego zbioru
	\item $IS$ $NULL$ - zawierania wartości NULL
\end{itemize}
\vspace{5mm}

\h{Operatory logiczne:}
W SQL mamy dostęp do operatorów logicznych:
\begin{itemize}
	\item AND
	\item OR
	\item NOT
\end{itemize}
\vspace{5mm}

\h{Komentarze w SQL:}
Aby wykomenotwać linię w SQL należy wprowadzić dwa znaki minusa po sobie.
\begin{lstlisting}[language=SQL]
-- komentarz jednolinijkowy
\end{lstlisting}

\noindent
Komentarz wieloliniowy otwiera się znakiem $'/*'$ a zamyka znakiem $'*/'$.
\begin{lstlisting}[language=SQL]
/* 
	Komentarz
	wieloliniowy
*/
\end{lstlisting}
\vspace{5mm}

\h{Typy danych w SQL:}
W języku SQL mamy dostęp do poniższych typów danych:
\begin{itemize}
	\item \italic{INTEGER} - liczby całkowite
	\item \italic{FLOAT} - liczba zmiennoprzecinkowa
	\item \italic{DECIMAL(precyzja, skala)} - liczba zmiennoprzecinkowa, której precyzję (ilość cyfr w liczbie) oraz skalę (ilość miejsc po przecinku) można dowolnie ustalić.
	\item \italic{CHAR(długość)} - ciąg znaków o stałej długości podanej 
	\item \italic{VARCHAR(maksymalna długość)} - przechowuje ciąg znaków o zmiennej długości (ale zawsze nie większej niż długość maksymalna)
	\item \italic{BOOLEAN} - wartości logiczne
	\item \italic{DATE} - daty
	\item \italic{TIME} - godziny
	\item \italic{TIMESTAMP} - daty i godziny
	\item \italic{BLOB} - dane binarne czyli obrazy, pliki dźwiękowe itd. 
\end{itemize} 
\vspace{5mm}

\h{Tworzenie nowej tabeli:}
Aby stworzyć nową tabelę należy użyć słowa kluczowego CREATE TABLE.

\begin{lstlisting}[language=SQL]
CREATE TABLE <nazwa tabeli> (
	<nazwa kolumny> <typ danych> [opcjonalna flaga kolumny],
	<nazwa kolumny> <typ danych> [opcjonalna flaga kolumny],
	...
	<nazwa kolumny> <typ danych> [opcjonalna flaga kolumny],
	[opcjonalne flagi tabeli]
);
\end{lstlisting}
\vspace{5mm}

\h{Usuwanie tabeli:}
Aby usunąć tabelę należy użyć słowa kluczowego DROP TABLE.
\begin{lstlisting}[language=SQL]
DROP TABLE <nazwa tabeli>;
\end{lstlisting}
\vspace{5mm}

\h{Dodawanie rekordu do tabeli:}
Dodać rekord do tabeli można za pomocą słowa kluczowego INSERT INTO. W komendzie specyfikuje się nazwy kolumn jakie mają zostać uzupełnione oraz wartości, którymi będą uzupełnione.
\begin{lstlisting}[language=SQL]
INSERT INTO <nazwa tabeli> 
( <nazwa kolumny>, ... , <nazwa kolumny> )
VALUES 
( <wartosc>, ... , <wartosc> );
\end{lstlisting}
\vspace{5mm}

\h{Polecenie DESCRIBE:}
Polecenie DESCRIBE służy do wyświetlenie informacji o strukturze tabeli i jej kolumnach. Zwraca ona dane:
\begin{itemize}
	\item \italic{Name} - nazwa kolumny
	\item \italic{Type} - typ danych przechowywanych w kolumnie
	\item \italic{Null} - informacja, czy kolumna może przyjmować wartość NULL
	\item \italic{Key} - informacja, czy kolumna jest kluczem (PRIMARY lub FOREIGN KEY)
	\item \italic{Default} - wartośc domyślna kolumny
\end{itemize}
\vspace{5mm}

\begin{lstlisting}[language=SQL]
DESCRIBE <nazwa tabeli>;
\end{lstlisting}
\vspace{5mm}

\h{Wyrażenie SELECT:}
Wyrażenie SELECT służy do pobierania danych z bazy danych. Jest to narzędzie do wyszukiwania, wybierania i pobierania określonych wartości z jednej lub wielu tabel w bazie danych.
\begin{lstlisting}[language=SQL]
SELECT <nazwa kolumny>, ... , <nazwa kolumny>
FROM <nazwa tabeli>
WHERE [warunek]
GROUP BY [nazwa kolumny], ... , [nazwa kolumny]
HAVING [warunek]
ORDER BY [kolumna];
\end{lstlisting}
\vspace{5mm}

\h{Warunek WHERE:}
Warunek WHERE służy do wybrania wyłącznie tych rekordów, które spełniają określone kryteria. W warunku występującym po warunku WHERE można zawrzeć wyrażenie logiczne zawierające operatory porównania, porównania liczbowego. Można również użyć operatorów logicznych AND, OR i NOT, które pozwalają łączenie wielu warunków w jedną całość.
\begin{lstlisting}[language=SQL]
	SELECT * -- gwiazdka oznacza wybranie wszytskich kolumn
	FROM produkty
	WHERE cena > 10;
\end{lstlisting}
\vspace{5mm}



\end{document}